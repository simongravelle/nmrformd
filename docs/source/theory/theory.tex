In essence, 




Note that associating intermolecular (respectively intramolecular) 
degrees of freedom to translational (respectively rotational) relaxation modes
is not rigorous. Rotational and translational motions are in general correlated \cite{henritzi13a, becher21a, becher22a},
particularly in the case of fluids confined into nanopores as,
for instance, rotational motions are strongly correlated to the position of the 
molecules in the pore. Nevertheless, here we will refer to intramolecular/intermolecular
degrees of freedom as rotational/translational as it helps to link the molecular motion 
to the NMR data.
The ensemble average is performed by a double summation over spin pair $i$ and
$j$ with $i \ne j$.
The functions ${\cal F}_{ij}^{(m)} (t)$ are dependent on the vector
$\boldsymbol{r}_{ij}$ which specifies the position of spin $j$ with respect to
spin $i$, and read
\begin{equation}
{\cal F}_{ij}^{(m)} (t) = \alpha_m 
\dfrac{1}{r_{ij}^3 (t)}
Y^{(m)}_2 (\theta_{ij} (t), \varphi_{ij} (t)),
\label{eq:Fij}
\end{equation}
where $Y^{(m)}_2$ are the normalized spherical harmonics with $\ell = 2$, and
\begin{equation}
\alpha_0 = \sqrt{\dfrac{16 \pi}{5}}, \; \alpha_1 = \sqrt{\dfrac{8 \pi}{15}}, \;
\alpha_2 = \sqrt{\dfrac{32 \pi}{15}}.
\label{eq:alpha}
\end{equation}
The spectral densities are obtained from the Fourier transforms of $G^{(m)}_\text{R,T} (t)$,
\begin{equation}
J^{(m)}_\text{R,T} (\omega) = \int_{-\infty}^\infty G^{(m)}_\text{R,T} (t) \; 
\mathrm e^{-i \omega t} \mathrm d t,
\label{eq:JRT}
\end{equation}
from which the relaxation time $T_{1, \text{R}, \text{T}}$ can be calculated as
\begin{equation}
\dfrac{1}{T_{1, \text{R}, \text{T}}} = K \left(J^{(1)}_\text{R,T} (\omega_0) 
+ J^{(2)}_\text{R,T} (2 \omega_0) \right),
\label{eq:T1RT}
\end{equation}
where $\omega_0 = \gamma B_0$ is the Larmor frequency.
The prefactor in Eq.\ \eqref{eq:T1RT} is given by 
\begin{equation}
K = \dfrac{3}{2}\left(\dfrac{\mu_0}{4 \pi}\right)^2 \hbar^2 \gamma^4 I (I+1),
\end{equation}
where $\mu_0$ is the vacuum permeability, $\hbar$ the reduced Planck constant, 
and $\gamma/ 2 \pi = 42.58$\,MHz/T is the gyro-magnetic ratio for $^1$H with spin $I = 1/2$.
The final expression for the relaxation time follows from the additivity,
\begin{eqnarray}
\dfrac{1}{T_{1}} & = & \dfrac{1}{T_{1, \text{R}}}+\dfrac{1}{T_{1, \text{T}}}.
%\dfrac{1}{T_{2}} & = & \dfrac{1}{T_{2, \text{R}}}+\dfrac{1}{T_{2, \text{T}}}.
\end{eqnarray}
Convergence tests were used to ensure that the simulation cut-off or system size
had no significant impact on the calculation of $T_1$, see the Supplemental
Material. 







































:cite:`agrawalViscousPeelingNanosheet2022` 

NMR relaxation time :math:`T_1` is calculated from 
the autocorrelation function of fluctuating magnetic 
dipole-dipole interactions (see |mcconnell1987| and |bloembergen1948| for more details),

.. math::

    G_\text{R,T} (t) = \dfrac{1}{N_\text{R,T}} \sum_{i \ne j}^{N_\text{R,T}} 
    \left< {\cal F}_{ij} (t + \tau) {\cal F}_{ij} (\tau)  \right>,

where :math:`\tau` is the lag time. Here :math:`N_\text{R}` and :math:`N_\text{T}`
correspond to partial ensembles for intramolecular and itermolecular interactions,
respectively, where R stands for rotational and T for translational (see |singer17a|). The ensemble average is performed by a 
double summation over spin pair :math:`i` and :math:`j` with :math:`i \ne j`.

The function :math:`{\cal F}_{ij} (t)` reads

.. math::
    
    {\cal F}_{ij} (t) = \sqrt{\dfrac{16}{5 \pi}} \dfrac{Y^0_2 (\theta_{ij} (t))}{r_{ij}^3 (t)},

where :math:`Y^0_2` is the normalised spherical harmonics with :math:`\ell = 2` and :math:`m = 0`,
:math:`r_{ij} (t)` the nuclear spin separation, and :math:`\theta_{ij} (t)` the polar angle
of the direction :math:`\textbf{r}_{ij}` with respect to laboratory axes (assuming that 
the applied static magnetic is parallel to :math:`\textbf{e}_z`).

NMR relaxation times are calculated using

.. math::

    \dfrac{1}{T_{1, \text{R}, \text{T}}} = K \left(J_\text{R,T} (\omega_0) + 4 J_\text{R,T} (2 \omega_0) \right),

were :math:`\omega_0 = \gamma B_0` is the Larmor frequency with :math:`\gamma$` the
gyro-magnetic ratio for :math:`^1`H with spin :math:`I = 1/2`, and 

.. math::

    K = \dfrac{1}{4}\left(\dfrac{\mu_0}{4 \pi}\right)^2 \hbar^2 \gamma^4 I (I+1),

where :math:`\mu_0` is the vacuum permeability.

.. |mcconnell1987| raw:: html

   <a href="https://www.cambridge.org/de/academic/subjects/physics/condensed-matter-physics-nanoscience-and-mesoscopic-physics/theory-nuclear-magnetic-relaxation-liquids?format=PB&isbn=9780521107716" target="_blank">mcconnell1987</a>

.. |bloembergen1948| raw:: html

   <a href="https://journals.aps.org/pr/abstract/10.1103/PhysRev.73.679" target="_blank">bloembergen1948</a>

.. |singer17a| raw:: html

   <a href="https://www.sciencedirect.com/science/article/pii/S1090780717300319" target="_blank">singer17a</a>
